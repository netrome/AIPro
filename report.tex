\documentclass{article}
\usepackage[utf8]{inputenc}

\usepackage[parfill]{parskip}
\usepackage[a4paper, total={6in, 10in}]{geometry}

\usepackage{amsmath}

\title{Team 2: Multi agent maze exploration}
\author{ Mårten Nilsson \texttt{marten3@kth.se}, Patrik Barkman
  \texttt{barkm@kth.se} \\ Axel Demborg \texttt{demborg@kth.se}, Elon Såndberg
  \texttt{elons@kth.se}}
\date{October 2016}

\begin{document}

\maketitle

\section{Abstract}

\section{Introduction}
%Classical planning... perfect information... exploration necessary... multiagent coordination...

Generating maps of certain areas of importance has long been a question of great interest in a
wide range of fields. Historically the explorers of the areas have been humans.
Though using humans for exploring certain areas is intractable. Both due to
limitations of human capabilities and risks associated with the exploration
task. For an example, during search and rescue operations in hostile
environments such as a building on fire one is not inclined to put more human
lives at risk than necessary. With the recent advances in AI and robotics,
another opportunity for these situations arises. The exploration could be
performed by specialized robots. This does not put any unneccessary human lives
at risk and could possibly lead to a more efficient operation.

One of the greatest challenges for automated multi agent exploration is how to
coordinate the agents in the unknown area. Depending on the limitations of the
robots in the terrain, different methods of addressing the problem arises. In
this report the focus is on some of the established algorithms in the field as
well as proposing some modifications. This study seeks to obtain data to provide
new knowledge of how these algorithms perform in a maze like environment where
conventional heuristics not always result in decent performance.

\section{Related work}
A considerable amount of literature has been published on multi agent
coordination motivated by search and rescue operations. Ferranit, Trigoni and
Levene proposed the \textit{Brick\&Mortar} algorithm in which the agents operate
on local information and communicate indirectly with each other by leaving
information tags at visited points \cite{ferranti2007brick}. Their algorithm is
inspired by the \textit{Ants} algorithm proposed by Koenig and Liu \cite{koenig2001terrain} and the
\textit{Multiple depth first search (MDFS)} algorithm which first occured in \cite{tarry1895probleme}. 

Traditionaly, the problem of multi agent exploration have been seen as a
distinct part of serarch and rescue operations. However, Becker, Blatt and
Szczerbicka proposed the flooding algorithm \cite{becker2013multi} in which the
exploration and rescue parts of the operation work in parallell. In their
algorithm the exploration agents act on local information and frequently report
back to a base of operations.

\section{Problem formulation}

Prims maze...

\section{Method(s)}

\subsection{Ants}

\subsection{Multiple Depth First Search (MDFS)}

\subsection{Deep Ants}

\subsection{Decentralized Search with Global Information}

\subsection{Centralized Search with Global Information}
In addition to the methods based on decentralized planning, attempts have been
made to device an algorithm with centralized planning using global information.
Such an algorithm could be realizable in environments where all agents are able
to communicate with a centralized system. Theoretically, this approach should
be the most effective, but also the most computationally demanding. This is
mainly due to the dimension of the search space grows exponentially with the
number of agents.

The method is based on the $A^*$ search algorithm. It is an informed search
where states are prioritized according to the cost function
%
$$f(x) = g(x) + h(x)$$
%
where $g(x)$ is the cost of traversing from the start to the state $x$ and
$h(x)$ is an heuristic that estimates the cost from $x$ to the goal. In this
case the start state consists of any possible placement of the agents in the
enviroment and in genernal the environment has only partly been explored by the
agents. The goal state is defined such that all the agents should be positioned
on a border to an undiscovered area of the environment. Since the agents will
have to travel different distance to reach such a border, as soon as one agent 
has found its way to the border its position is no longer varied during the 
remaining search.

The function $g(x)$ is defined as
%
$$g(x) = a * \begin{Bmatrix} \text{number of steps} \\ \text{from start to $x$}
\end{Bmatrix} + b * \begin{Bmatrix} \text{explored area at $x$} \\ - \\ \text{
explored area at start} \end{Bmatrix}$$
%
Since the goal is defined as above, the second term effectivley only matters
when the search is one step from the goal. This choice of cost function should
inclince the agents to prioritize goals that are close and approach that area in
such a way as to discover as large area as possible.

The heuristic $h(x)$ is defined as 
%
$$h(x) = c * \text{\{distance to nearest undiscovered area\} }$$
%
Since this underestimates the actual cost it is an admissible heuristic 
\cite{russell2003artificial}. The heuristic is crucial in order to guide the 
search in the right direction and consequently avoiding extensive computations.
This heuristic is in practice an estimate of the first term in $g(x)$ ---
unknown areas that are closer to the agent is prioritized.

In the case study explained below the coefficients $a = 10$, $b = 1$ and $c =
1$ were used.

\section{Case study}

\section{Performance measures}
\begin{itemize}
\item Exploration percentage vs number of iterations
\item Exploration time vs number of agents
\item Time per iteration vs number of iterations
\end{itemize}

\section{Results}

\section{Discussion}

\begin{itemize}
    \item Why not test Brick\&Mortar? -MDFS not optimal in our setting etc...
    \item Heuristic of Global Player should be better
\end{itemize}

\section{Conclusions}


\bibliographystyle{unsrt}
\bibliography{references}


\end{document}
