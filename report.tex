\documentclass{article}
\usepackage[utf8]{inputenc}

\usepackage[parfill]{parskip}
\usepackage[a4paper, total={6in, 10in}]{geometry}

\title{ Multi agent maze exploration: A review of aMAZEing algorithms }
\author{ Mårten Nilsson, Patrik Barkman, Axel Demborg, Elon Såndberg }
\date{October 2016}

\begin{document}

\maketitle

\section{Introduction}
%Classical planning... perfect information... exploration necessary... multiagent coordination...

Generating maps of certain areas of importance has long been a question of great interest in a
wide range of fields. Historically the explorers of the areas have been humans.
Though using humans for exploring certain areas is intractable. Both due to
limitations of human capabilities and risks associated with the exploration
task. For an example, during search and rescue operations in hostile
environments such as a building on fire one is not inclined to put more human
lives at risk than necessary. With the recent advances in AI and robotics,
another opportunity for these situations arises. The exploration could be
performed by specialized robots. This does not put any unneccessary human lives
at risk and could possibly lead to a more efficient operation.



\section{Problem formulation}

\section{Method(s)}

\section{Performance measures}
\begin{itemize}
\item Exploration percentage vs number of iterations
\item Exploration time vs number of agents
\item Time per iteration vs number of iterations
\end{itemize}

\section{Results}

\section{Discussion}

\section{Conclusions}



\end{document}

