\documentclass{article}
\usepackage[utf8]{inputenc}

\usepackage[parfill]{parskip}
\usepackage[a4paper, total={6in, 10in}]{geometry}

\title{Team 2: Multi agent maze exploration}
\author{ Mårten Nilsson \texttt{marten3@kth.se}, Patrik Barkman
  \texttt{barkm@kth.se} \\ Axel Demborg \texttt{demborg@kth.se}, Elon Såndberg
  \texttt{elons@kth.se}}
\date{October 2016}

\begin{document}

\maketitle

\section{Abstract}

\section{Introduction}
%Classical planning... perfect information... exploration necessary... multiagent coordination...

Generating maps of certain areas of importance has long been a question of great interest in a
wide range of fields. Historically the explorers of the areas have been humans.
Though using humans for exploring certain areas is intractable. Both due to
limitations of human capabilities and risks associated with the exploration
task. For an example, during search and rescue operations in hostile
environments such as a building on fire one is not inclined to put more human
lives at risk than necessary. With the recent advances in AI and robotics,
another opportunity for these situations arises. The exploration could be
performed by specialized robots. This does not put any unneccessary human lives
at risk and could possibly lead to a more efficient operation.

One of the greatest challenges for automated multi agent exploration is how to
coordinate the agents in the unknown area. Depending on the limitations of the
robots in the terrain, different methods of addressing the problem arises. In
this report the focus is on some of the established algorithms in the field as
well as proposing some modifications. This study seeks to obtain data to provide
new knowledge of how these algorithms perform in a maze like environment where
conventional heuristics not always result in decent performance.

\section{Related work}
A considerable amount of literature has been published on multi agent
coordination motivated by search and rescue operations. Ferranit, Trigoni and
Levene proposed the \textit{Brick\&Mortar} algorithm in which the agents operate
on local information and communicate indirectly with each other by leaving
information tags at visited points \cite{ferranti2007brick}. Their algorithm is
inspired by the \textit{Ants} algorithm proposed by Koenig and Liu \cite{koenig2001terrain} and the
\textit{Multiple depth first search (MDFS)} algorithm which first occured in \cite{tarry1895probleme}. 

Traditionaly, the problem of multi agent exploration have been seen as a
distinct part of serarch and rescue operations. However, Becker, Blatt and
Szczerbicka proposed the flooding algorithm \cite{becker2013multi} in which the
exploration and rescue parts of the operation work in parallell. In their
algorithm the exploration agents act on local information and frequently report
back to a base of operations.

\section{Problem formulation}

Prims maze...

\section{Method(s)}

\subsection{Ants}

\subsection{Multiple Depth First Search (MDFS)}

\subsection{Deep Ants}

\subsection{Centralized search}

\section{Case study}

\section{Performance measures}
\begin{itemize}
\item Exploration percentage vs number of iterations
\item Exploration time vs number of agents
\item Time per iteration vs number of iterations
\end{itemize}

\section{Results}

\section{Discussion}
Why not test Brick\&Mortar? -MDFS not optimal in our setting etc...

\section{Conclusions}


\bibliographystyle{unsrt}
\bibliography{references}


\end{document}

