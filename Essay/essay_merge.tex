
\documentclass{article}
\usepackage[utf8]{inputenc}
\usepackage[parfill]{parskip}
\usepackage[a4paper, total={6in, 10in}]{geometry}

\title{ The role and risks of artificial intelligence in our future society }
\author{ Mårten Nilsson }
\date{October 2016}

\begin{document}

\maketitle

\thispagestyle{empty}

\section{summary of discussion}
All parties in this discussion agree that automation of the majority of the human labour will inevitably happen within the next decades. In its core, we think this is beneficial for the human race and society at large. Some of us were concerned that this development may lead to unwanted effects concerning large waves of unemployment and depression. The current norm where unemplyed people are viewed as lazy and incompetent may be harmful to society when large portions of the population loses their jobs to artificial intelligence. The taboo of unemployment needs to be lifted in order to avoid civil unrest and depression which has previously been known to correlate with a high unemplyment rate. Our society will need to adjust so that these people can get the economical support to live, and only a small group of workers will provide for the society. One solution here could be to introduce some sort of citizen salary. 

No one in the group belevies in a massive AI apocalypse where the machines rise up against the human race. We believe however that machines may harm humans through accidents. One interesting topic we discussed concerning this is how society will look at accidents caused by non humans. The tolerance level for machine mistakes may be very low. For example if all cars would be transformed to self driving cars with the present technology, statistics could show that the traffic would be safer but accidents would still happen. If one of the self driving cars would kill a child, not a single individual would be blamed as in the situation of today but rather the wole entity of self driving cars would bear the responsability. 

We also discussed the robustness of a world run by AI. The ultimate goal of an automated society might lead us to a very fragile world, if we rely on robots to do all hard work for us. If the robotic industry is self-sustained, the need for humans to build the AI:s would perish. Some humans will probably have the skills necessary to sustain parts of society by pure interest at first, but when the reward for this is very low the risk is that vital knowledge is lost. This will probably not be noticed first but if a major solar flare wipes out a lot of major systems, the society might not have the ability to recover. Furthermore, another probable scenario where the vital systems might be disabled would be as a result of cyber warfare. In any case, the question of stability and security of the intelligent systems will likely be of uttermost importance.

In this discussion, the aim was to assess effects and risks of AI in our future society. The general opinion is that AI will bring many great things. Moreover, we are in agreement that we have to tread carefully, there are lots of things that could go wrong, both along the way to an AI governed world and when we are there. Proper care has to be taken now while we still have time define the course of our future.


\end{document}




